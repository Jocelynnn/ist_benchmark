
%%%%%%%%%%%%%%%%%%%%%%% file typeinst.tex %%%%%%%%%%%%%%%%%%%%%%%%%
%
% This is the LaTeX source for the instructions to authors using
% the LaTeX document class 'llncs.cls' for contributions to
% the Lecture Notes in Computer Sciences series.
% http://www.springer.com/lncs       Springer Heidelberg 2006/05/04
%
% It may be used as a template for your own input - copy it
% to a new file with a new name and use it as the basis
% for your article.
%
% NB: the document class 'llncs' has its own and detailed documentation, see
% ftp://ftp.springer.de/data/pubftp/pub/tex/latex/llncs/latex2e/llncsdoc.pdf
%
%%%%%%%%%%%%%%%%%%%%%%%%%%%%%%%%%%%%%%%%%%%%%%%%%%%%%%%%%%%%%%%%%%%


\documentclass[runningheads,a4paper]{llncs}

\usepackage{amssymb}
\setcounter{tocdepth}{3}
\usepackage{graphicx}
\usepackage{multirow}
\usepackage[table,xcdraw]{xcolor}
\usepackage{algorithm}
\usepackage{algorithmic}
\usepackage{amsmath}
\usepackage{natbib}


\usepackage{url}
\urldef{\mailsa}\path|{alfred.hofmann, ursula.barth, ingrid.haas, frank.holzwarth,|
\urldef{\mailsb}\path|anna.kramer, leonie.kunz, christine.reiss, nicole.sator,|
\urldef{\mailsc}\path|erika.siebert-cole, peter.strasser, lncs}@springer.com|
\newcommand{\keywords}[1]{\par\addvspace\baselineskip
\noindent\keywordname\enspace\ignorespaces#1}

\begin{document}

\mainmatter  % start of an individual contribution

% first the title is needed
\title{A general framework \\for evaluating interactive 
image segmentation algorithms}

% a short form should be given in case it is too long for the running head
\titlerunning{A general framework for evaluating interactive 
image segmentation algorithms}

% the name(s) of the author(s) follow(s) next
%
% NB: Chinese authors should write their first names(s) in front of
% their surnames. This ensures that the names appear correctly in
% the running heads and the author index.
%
\author{Bingjie Jiang%
\thanks{Please note that the LNCS Editorial assumes that all authors have used
the western naming convention, with given names preceding surnames. This determines
the structure of the names in the running heads and the author index.}%
\and Tonwei Ren}
%
% (feature abused for this document to repeat the title also on left hand pages)

% the affiliations are given next; don't give your e-mail address
% unless you accept that it will be published
\institute{Springer-Verlag, Computer Science Editorial,\\
Tiergartenstr. 17, 69121 Heidelberg, Germany\\
\mailsa\\
\mailsb\\
\mailsc\\
\url{http://www.springer.com/lncs}}



\toctitle{Lecture Notes in Computer Science}
\tocauthor{Authors' Instructions}
\maketitle


\begin{abstract}
The abstract should summarize the contents of the paper and should
contain at least 70 and at most 150 words. It should be written using the
\emph{abstract} environment.
\keywords{We would like to encourage you to list your keywords within
the abstract section}
\end{abstract}


\section{Introduction}
  \paragraph{}Interactive image segmentation has been extensively studied in the latest decade. Many state-of-the-art algorithms in this field have been proposed, starting from Boycov et. al\citep{boykov2001interactive},followed by Grabcut\citep{rother2004grabcut},Random Walker\citep{grady2006random},Bai and Sapiro \citep{bai2007geodesic} and \citep{gulshan2010geodesic}. However, when it comes to the evaluation of these algorithms, the comparison can hardly be objective due to different human interferences. As is often the case, interactive image segmentation algorithms are tested upon user scribbles provided by the specific author. In this way, the performance of segmentation result could heavily depend on certain batch of seeds selection, rendering the result not convincing enough when compared with other algorithms. 
 \paragraph{}This paper deals with the problem of evaluating interactive segmentation algorithms in an objective and comprehensive way. The contribution of this paper includes:
 \paragraph{} The remainder of this paper is organized as follows:
  
\section{Related Work}

******

\section{Dataset design}
\paragraph{}The dataset contains 96 images from publicly available Berkeley Segmentation Dataset\citep{martin2001database}

\section{User-interaction differences}



\bibliographystyle{plain}
\bibliography{references}

%\end{thebibliography}



\end{document}
